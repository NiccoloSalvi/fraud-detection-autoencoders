\section{Introduction}

In today's evolving digital economy, credit card fraud poses significant challenges for financial institutions and customers alike. According to the European Banking Authority (EBA) and the European Central Bank (ECB) report on Fraud Payments\footnote{2024 REPORT ON PAYMENT FRAUD, \url{https://www.ecb.europa.eu/press/intro/publications/pdf/ecb.ebaecb202408.en.pdf}}, fraudulent card-based transactions amounted to EUR 4.3 billion in 2022 and EUR 2.0 billion in the first half of 2023. Increasing transaction volumes, evolving fraudulent methods, and the inherent rarity of fraud incidents compared to legitimate transactions complicate detection efforts. Consequently, there is a growing reliance on machine learning methods to swiftly and accurately detect suspicious activities.

Fundamentally, fraud detection is framed as a binary classification task, distinguishing fraudulent from legitimate transactions by identifying anomalous patterns in data. This research leverages advanced machine learning techniques—autoencoders, attention-based long short-term memory (ALSTM) networks, gradient boosting (GB), and support vector machines (SVM)—to capture nuanced and complex transaction behaviors. Specifically, we replicate and extend the approach introduced by Mohammed Tayebi and Said El Kafhali in their paper \textbf{Combining Autoencoders and Deep Learning for Effective Fraud Detection in Credit Card Transactions}\cite{Tayebi2025}.

Autoencoders offer robust performance by learning latent representations of data, making them particularly suited for anomaly detection. Additionally, ALSTM networks excel at identifying significant temporal patterns within sequential transaction data through attention mechanisms. Building upon these techniques, we propose a comprehensive framework where an autoencoder initially models the fraud transaction distribution, generates synthetic fraud samples, and refines them via an SVM-based filtering strategy. Subsequently, these refined datasets feed into an ALSTM network integrated within a gradient boosting ensemble to enhance detection capabilities.

The contributions of this paper include: (i) developing a specialized pipeline tailored for severely imbalanced fraud detection scenarios; (ii) improving synthetic data realism through an SVM-filtered autoencoder strategy; and (iii) validating superior performance over conventional machine learning and basic deep learning methods.